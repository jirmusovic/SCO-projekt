


\documentclass[11pt, conference, a4paper]{IEEEtran}

\ifCLASSINFOpdf
  % \usepackage[pdftex]{graphicx}
  % declare the path(s) where your graphic files are
  % \graphicspath{{../pdf/}{../jpeg/}}
  % and their extensions so you won't have to specify these with
  % every instance of \includegraphics
  % \DeclareGraphicsExtensions{.pdf,.jpeg,.png}
\else
  % or other class option (dvipsone, dvipdf, if not using dvips). graphicx
  % will default to the driver specified in the system graphics.cfg if no
  % driver is specified.
  % \usepackage[dvips]{graphicx}
  % declare the path(s) where your graphic files are
  % \graphicspath{{../eps/}}
  % and their extensions so you won't have to specify these with
  % every instance of \includegraphics
  % \DeclareGraphicsExtensions{.eps}
\fi

\hyphenation{op-tical net-works semi-conduc-tor}

\def\abstractname{Abstrakt}
\def\IEEEkeywordsname{Klíčová slova}
\def\refname{Reference}

\begin{document}

\title{Zero day \' utok}


\author{\IEEEauthorblockN{Veronika Jirmusov\' a}
\IEEEauthorblockA{\textit{Fakulta informačních technologií} \\
\textit{Vysoké učení technické v~Brně}\\
Brno, 2024 \\
xjirmu00@stud.fit.vut.cz}}



% The paper headers
\markboth{Journal of \LaTeX\ Class Files,~Vol.~14, No.~8, August~2015}%
{Shell \MakeLowercase{\textit{et al.}}: Bare Demo of IEEEtran.cls for IEEE Journals}



% make the title area
\maketitle

% As a general rule, do not put math, special symbols or citations
% in the abstract or keywords.
\begin{abstract}
Zero-Day útoky jsou zranitelnosti v softwaru, které jsou objeveny a zneužity útočníky ještě předtím, než je vývojáři nebo bezpečnostní týmy odhalí a opraví. Tyto útoky představují obrovské riziko, protože nejsou proti nim dostupné žádné známé záplaty. Tento projekt se zaměřuje na analýzu Zero-Day útoků, jejich mechanismy, způsoby detekce a prevenci. Projekt bude zahrnovat přehled skutečných případů Zero-Day útoků a diskuzi o technologiích a metodách používaných k jejich minimalizaci.
\end{abstract}

% Note that keywords are not normally used for peerreview papers.
\begin{IEEEkeywords}
Zero-Day útok
\end{IEEEkeywords}


\IEEEpeerreviewmaketitle



\section{Úvod do Zero-Day zranitelností a jejich významu:}
Zero-Day zranitelnosti jsou jednou z největších hrozeb v oblasti kybernetické bezpečnosti. Termín "Zero-Day" se vztahuje na situaci, kdy je zranitelnost softwaru nalezena a zneužita útočníky dříve, než výrobce softwaru v\r ubec dostane nějaké informace o této zranitelnosti, a proto ještě neexistuje žádná oficiální záplata ani oprava. Tento název vychází z toho, že vývojáři mají "nula dní" na to, aby se chránili před touto hrozbou, což dává útočníkům šanci zneužít tuto zranitelnost.

Zero-Day zranitelnosti mohou vzniknout v různých typech systémů, jako jsou operační systémy, webové prohlížeče, mobilní aplikace, síťová zařízení a cloudové služby. Jejich široký potenciál ohrozit mnohé cíle činí tyto zranitelnosti obzvlášť vážnými.

\subsection{Význam a důsledky Zero-Day útoků}
Zero-Day útoky mají dalekosáhlé dopady na bezpečnost uživatelů, firem a států. Když útočníci najdou zranitelnost a využijí ji k vlastnímu prospěchu, mohou dosáhnout různých cílů, jako je neoprávněný přístup k citlivým datům, porušení integrity systému, získání úplné kontroly nad napadeným zařízením, nebo šíření malwaru. Tyto útoky často vedou k masivním bezpečnostním incidentům, únikům dat nebo vážným narušením služeb, což může mít podstatné finanční a reputační důsledky pro zasažené subjekty.

Například známý útok Stuxnet, který se zaměřil na průmyslové řídicí systémy, nebo zranitelnost EternalBlue, která vedla k šíření ransomwaru WannaCry, ukazují, jak vážné mohou být následky Zero-Day zranitelností. Tyto případy prokázaly, že Zero-Day útoky mohou být nejen nástrojem kybernetického zločinu, ale také prostředkem státem sponzorovaných operací.

\subsection{Rostoucí počet a sofistikovanost útoků}
V důsledku stále rostoucí digitalizace a vyšší závislosti na softwarových řešeních se zvyšuje jak počet, tak složitost Zero-Day útoků. Útočníci, ať už jednotlivci, organizované skupiny nebo státní subjekty, investují významné částky do objevování nových zranitelností. To vedlo k vzniku černého trhu, kde se Zero-Day exploity prodávají za vysoké ceny. V důsledku toho jsou tyto útoky častěji využívány k cíleným útokům na vládní instituce, kritickou infrastrukturu nebo komerční firmy.

Proto je zásadní, aby bezpečnostní experti, vývojáři a organizace soustředili své úsilí na výzkum, detekci a minimalizaci Zero-Day zranitelností. Zavedení účinných bezpečnostních opatření a rychlé reakce na zjištěné hrozby jsou rozhodující pro snížení rizika a ochranu uživatelů a systémů před potenciálními útoky.

\subsection{Struktura projektu}
V následujících kapitolách se zaměříme na analýzu reálných případů Zero-Day útoků, metody jejich detekce a mitigace. Cílem je nabídnout praktický pohled na problematiku a přispět k lepšímu pochopení toho, jak mohou být Zero-Day zranitelnosti identifikovány a minimalizovány.



\section{Příklady a analýza aktuálních Zero-Day útoků:}
\section{Technické detaily a analýza zranitelností}
\section{Způsoby detekce a prevence Zero-Day útoků:}
\section{Praktické ukázky detekce a mitigace}
\section{Doporučení a závěry}
% The very first letter is a 2 line initial drop letter followed
% by the rest of the first word in caps.
% 
% form to use if the first word consists of a single letter:
% \IEEEPARstart{A}{demo} file is ....
% 
% form to use if you need the single drop letter followed by
% normal text (unknown if ever used by the IEEE):
% \IEEEPARstart{A}{}demo file is ....
% 
% Some journals put the first two words in caps:
% \IEEEPARstart{T}{his demo} file is ....
% 
% Here we have the typical use of a "T" for an initial drop letter
% and "HIS" in caps to complete the first word.


Definice Zero-Day zranitelností a Zero-Day útoků.\\
Vysvětlení, proč jsou tyto útoky tak nebezpečné.\\
Přehled historických i aktuálních Zero-Day útoků a jejich dopad na bezpečnost (např. útoky na Microsoft, Adobe, nebo Google Chrome).\\
% You must have at least 2 lines in the paragraph with the drop letter
% (should never be an issue)


\subsection{Subsection Heading Here}
Subsection text here.

% needed in second column of first page if using \IEEEpubid
%\IEEEpubidadjcol

\subsubsection{Subsubsection Heading Here}
Subsubsection text here.


% An example of a floating figure using the graphicx package.
% Note that \label must occur AFTER (or within) \caption.
% For figures, \caption should occur after the \includegraphics.
% Note that IEEEtran v1.7 and later has special internal code that
% is designed to preserve the operation of \label within \caption
% even when the captionsoff option is in effect. However, because
% of issues like this, it may be the safest practice to put all your
% \label just after \caption rather than within \caption{}.
%
% Reminder: the "draftcls" or "draftclsnofoot", not "draft", class
% option should be used if it is desired that the figures are to be
% displayed while in draft mode.
%
%\begin{figure}[!t]
%\centering
%\includegraphics[width=2.5in]{myfigure}
% where an .eps filename suffix will be assumed under latex, 
% and a .pdf suffix will be assumed for pdflatex; or what has been declared
% via \DeclareGraphicsExtensions.
%\caption{Simulation results for the network.}
%\label{fig_sim}
%\end{figure}

% Note that the IEEE typically puts floats only at the top, even when this
% results in a large percentage of a column being occupied by floats.


% An example of a double column floating figure using two subfigures.
% (The subfig.sty package must be loaded for this to work.)
% The subfigure \label commands are set within each subfloat command,
% and the \label for the overall figure must come after \caption.
% \hfil is used as a separator to get equal spacing.
% Watch out that the combined width of all the subfigures on a 
% line do not exceed the text width or a line break will occur.
%
%\begin{figure*}[!t]
%\centering
%\subfloat[Case I]{\includegraphics[width=2.5in]{box}%
%\label{fig_first_case}}
%\hfil
%\subfloat[Case II]{\includegraphics[width=2.5in]{box}%
%\label{fig_second_case}}
%\caption{Simulation results for the network.}
%\label{fig_sim}
%\end{figure*}
%
% Note that often IEEE papers with subfigures do not employ subfigure
% captions (using the optional argument to \subfloat[]), but instead will
% reference/describe all of them (a), (b), etc., within the main caption.
% Be aware that for subfig.sty to generate the (a), (b), etc., subfigure
% labels, the optional argument to \subfloat must be present. If a
% subcaption is not desired, just leave its contents blank,
% e.g., \subfloat[].


% An example of a floating table. Note that, for IEEE style tables, the
% \caption command should come BEFORE the table and, given that table
% captions serve much like titles, are usually capitalized except for words
% such as a, an, and, as, at, but, by, for, in, nor, of, on, or, the, to
% and up, which are usually not capitalized unless they are the first or
% last word of the caption. Table text will default to \footnotesize as
% the IEEE normally uses this smaller font for tables.
% The \label must come after \caption as always.
%
%\begin{table}[!t]
%% increase table row spacing, adjust to taste
%\renewcommand{\arraystretch}{1.3}
% if using array.sty, it might be a good idea to tweak the value of
% \extrarowheight as needed to properly center the text within the cells
%\caption{An Example of a Table}
%\label{table_example}
%\centering
%% Some packages, such as MDW tools, offer better commands for making tables
%% than the plain LaTeX2e tabular which is used here.
%\begin{tabular}{|c||c|}
%\hline
%One & Two\\
%\hline
%Three & Four\\
%\hline
%\end{tabular}
%\end{table}


% Note that the IEEE does not put floats in the very first column
% - or typically anywhere on the first page for that matter. Also,
% in-text middle ("here") positioning is typically not used, but it
% is allowed and encouraged for Computer Society conferences (but
% not Computer Society journals). Most IEEE journals/conferences use
% top floats exclusively. 
% Note that, LaTeX2e, unlike IEEE journals/conferences, places
% footnotes above bottom floats. This can be corrected via the
% \fnbelowfloat command of the stfloats package.





% if have a single appendix:
%\appendix[Proof of the Zonklar Equations]
% or
%\appendix  % for no appendix heading
% do not use \section anymore after \appendix, only \section*
% is possibly needed

% use appendices with more than one appendix
% then use \section to start each appendix
% you must declare a \section before using any
% \subsection or using \label (\appendices by itself
% starts a section numbered zero.)
%


\appendices
\section{Proof of the First Zonklar Equation}
Appendix one text goes here.

% you can choose not to have a title for an appendix
% if you want by leaving the argument blank
\section{}
Appendix two text goes here.


% use section* for acknowledgment
\section*{Acknowledgment}


The authors would like to thank...


% Can use something like this to put references on a page
% by themselves when using endfloat and the captionsoff option.
\ifCLASSOPTIONcaptionsoff
  \newpage
\fi



% trigger a \newpage just before the given reference
% number - used to balance the columns on the last page
% adjust value as needed - may need to be readjusted if
% the document is modified later
%\IEEEtriggeratref{8}
% The "triggered" command can be changed if desired:
%\IEEEtriggercmd{\enlargethispage{-5in}}

% references section

% can use a bibliography generated by BibTeX as a .bbl file
% BibTeX documentation can be easily obtained at:
% http://mirror.ctan.org/biblio/bibtex/contrib/doc/
% The IEEEtran BibTeX style support page is at:
% http://www.michaelshell.org/tex/ieeetran/bibtex/
%\bibliographystyle{IEEEtran}
% argument is your BibTeX string definitions and bibliography database(s)
%\bibliography{IEEEabrv,../bib/paper}
%
% <OR> manually copy in the resultant .bbl file
% set second argument of \begin to the number of references
% (used to reserve space for the reference number labels box)
\begin{thebibliography}{1}

\bibitem{IEEEhowto:kopka}
H.~Kopka and P.~W. Daly, \emph{A Guide to \LaTeX}, 3rd~ed.\hskip 1em plus
  0.5em minus 0.4em\relax Harlow, England: Addison-Wesley, 1999.

\end{thebibliography}

% biography section
% 
% If you have an EPS/PDF photo (graphicx package needed) extra braces are
% needed around the contents of the optional argument to biography to prevent
% the LaTeX parser from getting confused when it sees the complicated
% \includegraphics command within an optional argument. (You could create
% your own custom macro containing the \includegraphics command to make things
% simpler here.)
%\begin{IEEEbiography}[{\includegraphics[width=1in,height=1.25in,clip,keepaspectratio]{mshell}}]{Michael Shell}
% or if you just want to reserve a space for a photo:

\begin{IEEEbiography}{Michael Shell}
Biography text here.
\end{IEEEbiography}

% if you will not have a photo at all:
\begin{IEEEbiographynophoto}{John Doe}
Biography text here.
\end{IEEEbiographynophoto}

% insert where needed to balance the two columns on the last page with
% biographies
%\newpage

\begin{IEEEbiographynophoto}{Jane Doe}
Biography text here.
\end{IEEEbiographynophoto}

% You can push biographies down or up by placing
% a \vfill before or after them. The appropriate
% use of \vfill depends on what kind of text is
% on the last page and whether or not the columns
% are being equalized.

%\vfill

% Can be used to pull up biographies so that the bottom of the last one
% is flush with the other column.
%\enlargethispage{-5in}



% that's all folks
\end{document}


